\documentclass[11pt,a4paper,openany]{book}
\usepackage{graphicx}
\usepackage{setspace}	%double spacing for text, single for captions, footnotes, etc.
%\usepackage{hypernat} 	%substitut de cite que permet fer hyperlinks
\usepackage{natbib}		% substituye a 'hypernat' que funciona en Windows.
\usepackage[english]{babel}
\usepackage[utf8]{inputenc}
\usepackage{color}
\usepackage[dvipsnames, table]{xcolor}
\usepackage{hhline} 		% extended styles for tables
\usepackage{multirow}
\usepackage{subfigure}
\usepackage{acronym}
\usepackage{hyperref}
\usepackage{amsmath,amsmath,amssymb} 
\usepackage{fancyhdr}
\usepackage{epsfig, amsmath}
\usepackage{algorithm}
\usepackage{algorithmic}
\usepackage{pgfgantt}
\usepackage{placeins}

\usepackage{tcolorbox}
\usepackage{tabularx}

\usepackage{epigraph}

\RequirePackage{booktabs}

% general settings
\hypersetup{
	linktocpage=true,
	colorlinks=true,
	linkcolor=blue,
	citecolor=blue,
}
\definecolor{Hgray}{gray}{0.6}

\newenvironment{definition}[1][Definition]{\begin{trivlist}
\item[\hskip \labelsep {\bfseries #1}]}{\end{trivlist}}


%
% Nuevo comando para resaltar nombres de archivo.
\newcommand{\filename}[1]{\texttt{#1}}

%
% Nuevo comando para URLs.
\newcommand{\myurl}[2]{\href{#1}{#2}\footnote{\url{#1}}}

%
% Imagenes
\newcommand{\imagen}[3][0.8]{
	\begin{figure}[!ht]
		\centering
		\includegraphics[width=#1\textwidth]{#2}
		\caption{#3}\label{fig:#2}
	\end{figure}
	%\FloatBarrier
}

%
% Tabla
\newcolumntype{C}{>{\arraybackslash}X} % centered "X" column

%
% Red color for to be done or to be replaced things in document
\newcommand{\TBD}[1]{\textcolor{BrickRed}{#1}}

%
%Paragraph formatting
\setlength{\topmargin}{0cm}
\setlength{\oddsidemargin}{0cm}
\setlength{\evensidemargin}{0cm}
\setlength{\textheight}{23cm}
\setlength{\textwidth}{17cm}
\setlength{\headheight}{1cm}

% indica que las 'sub-sub-sections' sean numeradas y aparezcan en el indice
\setcounter{secnumdepth}{3}
\setcounter{tocdepth}{3}

% settings for code
\renewcommand{\algorithmicrequire}{\textbf{Entrada: }}
\renewcommand{\algorithmicensure}{\textbf{Salida: }}

%%%%%%%%%%%%
% DOCUMENT %
%%%%%%%%%%%%
\begin{document}

% portada
\newpage
\thispagestyle{empty}

\baselineskip 2em

%\vspace*{1cm}

\centerline{\includegraphics[width=0.6\textwidth]{images/UOC-logo}}
\begin{center}
\textsc{Universitat Oberta de Catalunya (UOC) \\
 Masters in Data Science\\}

%\centerline {\pic{UOC}{4cm}}

\vspace*{1.5cm}

\textsc{\Large MASTER'S THESIS}

\vspace*{0.5cm}

\textsc{\large Area: Medicine Area (TFM-Med)}


%\textbf{\Huge VirtualTechLab Model: }

\vspace*{2.0cm}

\textbf{\Large Deep Convolutional Autoencoders for reconstructing magnetic resonance images of the healthy brain}

\vspace{2.5cm}
\baselineskip 1em

\baselineskip 2em
-----------------------------------------------------------------------------\\
Author:     Adrián Arnaiz Rodríguez\\
Tutor:      Baris Kanber\\
TFM Professor: Ferran Prados Carrasco\\
-----------------------------------------------------------------------------\\
\vspace*{1.5cm}
Barcelona, \today

\end{center}

\newpage
\pagestyle{empty}
\hfill

\newpage
% abstract
\pagenumbering{roman} 
\setcounter{page}{1} 
\pagestyle{plain}

%%%%%%%%%%%%%%%%
%%% CREDITOS %%%
%%%%%%%%%%%%%%%%
\chapter*{Créditos/Copyright}

\vspace{1cm}

\begin{figure}[ht]
    \centering
	\includegraphics[scale=1]{images/license.png}
\end{figure}

Este obra está bajo una \href{http://creativecommons.org/licenses/by-nc-sa/3.0/es/}{licencia de Creative Commons \textit{Reconocimiento, NoComercial, CompartirIgual} 3.0 España}.

\begin{figure}[ht]
    \centering
	\includegraphics[scale=1]{images/license.png}
\end{figure}

This work is licensed under a \href{http://creativecommons.org/licenses/by-nc-sa/3.0/es/deed.en}{Creative Commons Attribution-NonCommercial-ShareAlike 3.0 Spain License}.

The \myurl{https://github.com/AdrianArnaiz/Brain-MRI-Autoencoder}{official code repository of this Master's Thesis} is licensed under MIT license.

%%%%%%%%%%%%%
%%% FICHA %%%
%%%%%%%%%%%%%
\chapter*{FICHA DEL TRABAJO FINAL}

\begin{table}[ht]
	\centering{}
	\renewcommand{\arraystretch}{2}
	\begin{tabular}{r | l}
		\hline
		Título del trabajo: & \vtop{\hbox{\strut Deep Convolutional Autoencoder} \hbox{\strut for control brain MRI:}\hbox{\strut Development and Applications}}\\
		\hline
        Nombre del autor: & Adrián Arnaiz Rodríguez\\
		\hline
        Nombre del colaborador/a docente: & Baris Kanber\\
		\hline
        Nombre del PRA: & Ferrán Prados Carrasco\\
		\hline
        Fecha de entrega (mm/aaaa): & 01/2021\\
		\hline
        Titulación o programa: & Máster en Ciencia de Datos\\
		\hline
        Área del Trabajo Final: & Area Medicina (TFM-Med)\\
		\hline
        Language: & English\\
		\hline
        Keywords & Deep Learning, Brain MRI, Autoencoder\\
		\hline
	\end{tabular}
\end{table}

%%%%%%%%%%%%%%%%%%%
%%% DEDICATORIA %%%
%%%%%%%%%%%%%%%%%%%
\chapter*{Dedication/Cite}

\TBD{To Be Done.}


%%%%%%%%%%%%%%%%%%%
%%% Agradecimientos %%%
%%%%%%%%%%%%%%%%%%%
\chapter*{Acknowledgment}

\TBD{To Be Done.
Si se considera oportuno, mencionar a las personas, empresas o instituciones que hayan contribuido en la realización de este proyecto.}

%%%%%%%%%%%%%%%%
%%% RESUMEN  %%%
%%%%%%%%%%%%%%%%
\chapter*{\centering Abstract}
\addcontentsline{toc}{chapter}{Abstract}

\onehalfspacing

\begin{quote}
{The analysis of brain MRI is critical for a proper diagnosis and treatment of neurological diseases. Improvements in this field lead to better health quality. Numerous branches can be still enhanced due to the nature of MRI recompilation: disease detection and segmentation, data augmentation, improvement in data collection, or image enhancement are some of them.

For several years, many approaches have been taken to address this. Machine Learning and Deep Learning emerge as very popular approaches to solve problems. Several kinds of data mining problems (supervised, unsupervised, dimension reduction, generative models, etc) and algorithms can be applied to the problem-solving of MRI. Besides, new emerging deep learning architectures for another kind of image task can be helpful. New types of convolution, autoencoders or generative adversarial networks are some of them.

Therefore, the purpose of this work is to apply one of these new techniques to T1 weighted brain MRI. We will develop a Deep Convolutional Autoencoder, which can be used to help some problems with neuroimaging. The input of the Autoencoder will be control T1WMRI and aims to return the same image, with the problematic that, inside its architecture, the image travels through a lower-dimensional space, so the reconstruction of the original image becomes more difficult. Thus, the Autoencoder represents a normative model.

This normative model will define a distribution (or normal range) for the neuroanatomical variability for the illness absence. Once trained with these control images, we will discuss the potential application of the autoencoder like noise reducer or disease detector.}
\end{quote}

%{http://www.ece.cmu.edu/~koopman/essays/abstract.html}
%{http://www.ece.cmu.edu/~koopman/essays/abstract.html}

\textbf{Keywords}: Deep Learning, Brain MRI, Deep Convolutional Autoencoder, Image denoising.

\clearpage

\chapter*{\centering Resumen}

\addcontentsline{toc}{chapter}{Resumen}

\begin{quote}
{El análisis de las resonancias magnéticas cerebrales es fundamental para un diagnóstico y tratamiento adecuados de las enfermedades neurológicas. Se pueden mejorar ámbitos del análisis debido a la naturaleza de la recopilación de resonancias: detección y segmentación de enfermedades, aumento de datos, mejora en la extracción o mejora de imágenes.

El aprendizaje automático y el aprendizaje profundo surgen como nuevas alternativas populares para resolver estos problemas. Se pueden aplicar varios enfoques de minería de datos y algoritmos para la resolución de problemas relacionados con la neuroimagen (supervisados, no supervisados, reducción de dimensionalidad, modelos generativos, etc.). Además, las nuevas arquitecturas emergentes de aprendizaje profundo, desarrolados para otro tipo de tareas de imagen, pueden ser útiles. Algunas de ellas son nuevos tipos de convolución, autoencoders o GAN.

Por lo tanto, el propósito de este trabajo es aplicar una de estas nuevas técnicas a resonancias cerebrales tipo T1. Desarrollaremos un Autoencoder convolucional profundo, que puede usarse para ayudar con algunos problemas de neuroimagen. La entrada del Autoencoder será el imágenes de control T1WMRI y tendrá como objetivo devolver la misma imagen, con la problemática de que, dentro de su arquitectura, la imagen viaja por un espacio de menor dimensión, por lo que la reconstrucción de la imagen original se vuelve más difícil. El autoencoder representa un modelo normativo.

Este modelo normativo definirá una distribución (o rango normal) para la variabilidad neuroanatómica para la ausencia de enfermedad. Una vez entrenado con imágenes de control, discutiremos la aplicación potencial del Autoencoder como reductor de ruido o detector de enfermedades.

}
\end{quote}
\textbf{Keywords}: Aprendizaje profundo, Imágenes cerebrales de resonancias magnéticas, Autoencoder convolucional profundo, eliminación de ruido de imágenes.



\clearpage
\newpage

\pagestyle{fancy}
\renewcommand{\chaptermark}[1]{ \markboth{#1}{}}
\renewcommand{\sectionmark}[1]{\markright{ \thesection.\ #1}}
\lhead[\fancyplain{}{\bfseries\thepage}]{\fancyplain{}{\bfseries\rightmark}}
\rhead[\fancyplain{}{\bfseries\leftmark}]{\fancyplain{}{\bfseries\thepage}}
\cfoot{}

% indice
\cleardoublepage
\phantomsection
\addcontentsline{toc}{chapter}{Content}
\tableofcontents
% listado de figuras
\cleardoublepage
\phantomsection
\addcontentsline{toc}{chapter}{List of Figures}
\listoffigures
% listado de tablas
\cleardoublepage
\phantomsection
\addcontentsline{toc}{chapter}{List of Tables}
\listoftables

\thispagestyle{empty}

\pagenumbering{arabic}

\pagestyle{fancy}
\renewcommand{\chaptermark}[1]{ \markboth{#1}{}}
\renewcommand{\sectionmark}[1]{\markright{ \thesection.\ #1}}
\lhead[\fancyplain{}{\bfseries\thepage}]{\fancyplain{}{\bfseries\rightmark}}
\rhead[\fancyplain{}{\bfseries\leftmark}]{\fancyplain{}{\bfseries\thepage}}
\cfoot{}

%http://kb.mit.edu/confluence/pages/viewpage.action?pageId=3907092
\onehalfspacing
\setlength{\parskip}{1em}
\rmfamily

% capitulos del documento
\chapter{Introduction}
\label{chapter:introduccion}

\textit{Not Final Note: \TBD{This color ('TBD' tag in latex) are going to be used in non-final versions of the memory to highlight the non-final sentences or things to be changed in the final version}}.

In this chapter we will introduce the main problematic of the project, basing it on its non-solved tasks and relevance.

%%% SECTION
\section{Problem overview and relevance}

\subsection{MRI general problems}

Neuroimaging in psychiatry allows studying the morphological features of the human brain. With the objective of improving the detection systems, diagnostic and treatment, correlations between the morphological features and the neuropsychiatric disorders can be addressed in order to achieve that \cite{abou2006neuroimaging}. If we improve brain magnetic resonance image analysis \ref{fig:figs/MRI_image.jpg}, we will improve the detection systems and treatments for neurological diseases, so the social relevance of the field is very important.

\imagen{figs/MRI_image.jpg}{Brain MRI examples \cite{fotomri}}

The relevance of the project is also shown that there are a lot of branches in which neuroimaging analysis can improve. Machine learning, and more recently Deep Learning and Computer Vision, has irrupted in this field for helping in some tasks: 

\begin{itemize}
    \item \textbf{Disease detection: segmentation and classification}. There are still several problems to solve in classification problem. Brain MRIs are high dimensional, so we have to recruit big amount of images to properly develop a Machine Learning model that be able to achieve high accuracy. It is very difficult to recruit a large number of images, especially disease images. Even if it performs well, machine learning algorithms have been criticized due to the difficult of extract a clear knowledge of them (black-boxes).  \cite{myronenko20183d}. So a experimental approach is disease detection based on outliers from a normative model Patients with pathologies will be outliers in the distribution build by the normative model (they will be out of normal range defined by the normative model) \cite{marquand2016normative} \cite{mourao2011outlier}.
    \item \textbf{Data Augmentation}. Lack of data problem can be addressed by Data Augmentation techniques, which look for improve our MRI datasets \cite{GanDataAugment2018}.
    \item \textbf{Improvement of data collection}. Recently high-impact \myurl{https://fastmri.org/}{FastMRI} release from Facebook for improving the speed in MRI scans \cite{fastmri}.
    \item \underline{\textbf{Image enhancement}}. Clinical evaluation is critical for good disease treatment. Experts and algorithms need good quality images to carry out their tasks. This is the main problem we want to address, so we will explain it deeper in the document \cite{tamada2020review} \cite{myronenko20183d}. 
\end{itemize}

\subsection{MRI Image enhancement}

\textbf{We are going to focus on the problem of image enhancement, specifically the problem of noise removal}, see \ref{fig:figs/Denoised_MRI.jpeg}. If we achieved good performance in this task, we would research about how to apply this solution to disease detection or data augmentation.

\imagen{figs/Denoised_MRI.jpeg}{Noised and Denoised MRI \cite{fotoDenoisedMri}}

Magnetic resonance images are collected with MR scans and the scan process, even though it is improved continually, adds some failures to the MR image. MR images have some random \textbf{noise} and \textbf{artifacts} due to this fact \cite{artifacts86}. This noise and artifacts are present in the image due to different reasons: hardware-reasons (magnetic fields, etc.), body motion during scanning, thermal noise, weak signal intensity (which causes low signal-to-noise ratio), etc. The difference between noise and artifact is that noise can hide the characteristics of an image, whereas artifacts appear to be characteristic but are not. If the 'problem' is structured, it is probably an artifact, while if it is random, it is probably noise.

\subsection{Noise and artifact reduction with Deep Learning}

To address this problem many approaches have been done, all of them with some disadvantages. Advanced filtering methods \cite{filtermeth12} or retrospective correction approaches have been proposed, but, with the rise of \textbf{Deep Learning}, other methods have been proposed that take advantage of this approach. Deep Learning is very powerful in high-dimensional spaces and non-linear problems, so it can make a better job in feature extraction or information compression. Therefore, it will have good performance with images, where the \textbf{underlying structure of the images will be captured and foreign elements such as noise or artifacts can be eliminated}. 

There are some approaches to reduce the noise and artifacts in MRI images. An recent and outstanding review is made by D.Tamada \cite{tamada2020review}. In this paper D.Tamada summarize Deep Learning Architectures and applications to MRI. We notice the big relevance of denoiser MRI Deep Learning methods in this review. Although there are many methods, we will focus in brain MRI denoisers. 

There are some Deep Learning Architectures to address in the problem of denoise an brain MRI: Single-scale CNN, Denoising CNN, Autoencoders, and Gan-based architectures. \textbf{We will choose the Autoencoder Architecture for this project}.

\textbf{Autoencoders} \cite{autoencoder} encode the input into a lower-dimensional space. It extracts important information from the higher-dimensional space, encodes it, and then decodes it to reconstruct the higher-dimensional spacer from the lower one. \TBD{This architecture will be deeply explained in \ref{chapter:theory} in further stages of the project.}



\subsection{Our approach}

\begin{tcolorbox}
In this project, we will design a \textbf{Deep Learning autoencoder for reconstructing brain magnetic resonance images removing noise. In other words, we will train a autoencoder with disease-free neuroimaging data and, with this trained autoencoder, we could define a distribution (or normal range) for the neuroanatomical variability for the illness absence with the main purpose of removing noise. Once trained, the autoencoder should be able to encode a input image and decode it removing noise.}
\end{tcolorbox}

\textbf{If the main objective of the project is completed, we could research the application oh this model in data augmentation and disease detection fields}. In the case of \textbf{data augmentation}, we will then attempt to reconstruct magnetic resonance images from patients with brain pathologies, with a further view to using the autoencoder to generate 'pathology-free' versions of the said images. In the case of \textbf{disease detection}, we could take an approach like the one in \cite{pinaya2019} (creating a measure for the difference between input and output image and classify it as healthy or control based on this measure). Patients with pathologies will be outliers in the distribution build by the autoencoder (they will be out of normal range defined by the autoencoder) \cite{marquand2016normative} \cite{mourao2011outlier}. This assumption of patients as outliers (based in \cite{mourao2011outlier}) is used in \cite{pinaya2019} for abnormal brain structure detection.

Of course, we will need \textbf{data}. For this project, we will use \textbf{T1-weighted MRI images of control subjects} (no disease). As the project is of fairly limited length, we won't need to detect disease as a principal objective, but only learn to reconstruct normal MRI images, so we won't need pathology images by the moment. In addition, we will investigate whether our method of reconstruction can filter out noise and/or artefacts. So, in essence, we will have $n$ 3D MRI volumes ($n$ can be any number greater than 100) from healthy subjects, we will preprocess it (data augmentation based on adding noise, remove parts...), and train our model to reconstruct the source MRI volumes. We will have access magnetic brain resonance images from control subjects for training the autoencoder. This data is arranged from different sources. \TBD{Concrete sources of data images used in the project will be discussed in further stages but we have some clear options in this moment}:

\begin{itemize}
    \item The \myurl{https://brain-development.org/ixi-dataset/}{IXI dataset} (\TBD{Most likely to be chosen}).
    \item Other data sources such as \myurl{https://openneuro.org}{Open Neuro}.
\end{itemize}



\section{Personal motivation}

My personal motivation to carry out the project arises from several factors. My first steps in the world of Machine Learning were in the last year of my career at the University of Burgos. I was lucky enough to collaborate last year with the \myurl{http://admirable-ubu.es/}{ADMIRABLE} research department. The project that I did (in which we continue working) was on the \myurl{https://github.com/AdrianArnaiz/TFG-Neurodegenerative-Disease-Detection}{use of biomarkers extracted from the voice for the construction of classifiers that detect Parkinson's}. The project include topics like \textbf{signal processing}, \textbf{supervised learning}, \textbf{unsupervised learning} and \textbf{transfer learning}. The project was very successful and we had a lot of impact at that time. We are currently in the process of meeting with the Burgos hospital to continue developing the model and the application (project and impact recompilation in Github).

This project has fully opened me the doors the world of artificial intelligence and machine learning, which is a field of knowledge that I love. I have always liked math, problem-solving and since I started my career I love programming. Therefore, I find this field the ideal that aligns with my tastes and interests. As i said before, I have done lot of jobs with supervised learning or data analysis, but only with tabular data sets or text-datasets, so I wanted to break in the world of image processing and Deep Learning.

Then I worked half year in Ernst and Young, developing Machine Learning systems for RPA tasks (classify emails at Telefónica, Chatbot for Maxium or Fuzzy Name Matching for Xunta de Galicia).

In addition, I believe that the application of AI to medicine is one of the fields that may be of greater general interest to society. By advancing in the speed and quality of medical diagnoses and treatments, it will be possible to achieve health of higher quality, speed and accessibility for all. Also, computer vision and deep learning have helped to achieve the big advances in this field nowadays. 


\section{State of art: related works}
\textit{\TBD{In later stages of the project, the resume of related works and state of art will be replaced in other place of the document.}}

The world relevance and impact of this problem is also shown in the related articles of this subject. The state of art of Deep Learning Autoencoders applied to brain MRI shows the relevance of this field. \TBD{In this stage of the project (\today), we have deeply read 3 papers for building the first bricks and identify the goals and methodology of the project}.

\begin{itemize}
    \item Walter Pinaya, 2018 \cite{pinaya2019}:
    
    Classic methods and approaches based on sMRI can't get a good performance in abnormal brain structural detection because neuroanatomical alterations in neuropsychiatric disorders are commonly subtle and spatially distributed. Another approach based on Machine Learning methods could improve performance. ML algorithms are sensitive to these subtle characteristics. The downside of this road is the need for a large amount of image data (control and disease) and that the models are black-boxes with no information on the critical characteristics used for the decision. With this Deep Autoencoder, they put this matter up for discussion.
    
    In this project, they address the problem of creating a normative model using a deep autoencoder trained with control subjects. With this autoencoder defining a distribution for control patients, they define a deviation metric to measure the neuroanatomical deviation in patients. Patients with some disorder should be outliers in this distribution.
    
    Architecture and technique used in the experiment:
    \begin{itemize}
        \item Semi-supervised autoencoder: reconstruction of the image and prediction of age and sex.
        \item 3 hidden layers with SELUs activation function.
        \item Output layer with Linear activation function.
        \item Loss function: MSE from reconstructed and original image + cross-entropy for age + cross-entropy for years + Unsupervised cross-covariance.
        \item 2000 epochs and 2000 iterations.
        \item Adam with adaptative learning rate.
        \item 64 samples mini-batches.
        \item Transformation of input image:
        \begin{itemize}
            \item Add Gaussian noise (0, 0.1).
            \item Feature scaling (normalization).
        \end{itemize}
        \item One-hot encoding for \texttt{sex} and \texttt{year} labels.
    \end{itemize}
    
    \item \TBD{We made a recompilation of papers reviewed in Table 1 of D. Tamada, 2020 \cite{tamada2020review} and by our own, with some removals and additions based in our goal of the project (see \ref{table:paper_overview}). In further stages we will explain deeply some papers from the table \ref{table:paper_overview}}. From the table of D. Tamada we only obtain 1 autoencoder study \cite{bermudez2018t1autoencoder}, 3 sCNN and DnCNN approaches \cite{kidoh2019scnnt1} \cite{ganHR3d} \cite{dncnnnoise2noise} and 1 GAN study. The other 5 papers have been compiled by our own (Autoencoder based: \cite{pinaya2019} \cite{gondara2016medicalautoencoder} \cite{superresolution} \cite{fuzzyautoencoder}, GAN-Autoencoder-based: \cite{wganautoencoder}).
\end{itemize}    

\begin{table}[!ht]
    \setlength\extrarowheight{2pt} % for a bit of visual "breathing space"
    \rowcolors {2}{gray!15}{}
    \begin{tabularx}{\textwidth}{C C C}
    \hline
        \textbf{Purpose} & \textbf{Year, Authors} & \textbf{Network} \\
        \hline
        
        \rowcolor{orange!10}
        \multicolumn{3}{c}{\textbf{Autoencoders}}\\
        
        \hline
        
        Identify brain abnormal structural patterns & 2018, W. Pinaya, et al \cite{pinaya2019} & \textbf{Semi-supervised autoencoder for HCP Dataset} \\
        
        Denoising for T1 weighted brain MRI & 2018, C. Bermudez, et al \cite{bermudez2018t1autoencoder} & \textbf{Autoencoder with skip connections} \\
        
        Medical image denoise & 2016, L. Gondara, et al \cite{gondara2016medicalautoencoder} &\textbf{Convolutional denoising autoencoder} \\
        
        General image denoising and super resolution & 2016, XJ. Mao et all \cite{superresolution} [Code available] & \textbf{Convolutional auto-encoders with symmetric skip connections} \\
        
        Brain MRI denoise & 2019, N. Chauhan et al \cite{fuzzyautoencoder} & \textbf{Convolutional denoising autoencoder with Fuzzy Logic filters} \\

        \hline
        \rowcolor{orange!10}
        \multicolumn{3}{c}{\textbf{sCNN and DnCNN}}\\
        \hline

        Denoising for T1, T2 and FLAIR brain images & 2018, M. Kidoh, et al \cite{kidoh2019scnnt1} & Single-scale CNN with DCT \\
        
        Motion artifact reduction for brain MRI & 2018, P. Johnson, et al \cite{scnnmotion} & Single-scale CNN\\
        
        Denoising for multishot DWI & 2020, M Kawamura et al \cite{dncnnnoise2noise} & DnCNN with Noise2Noise \\
        
        \hline
        \rowcolor{orange!10}
        \multicolumn{3}{c}{\textbf{GAN}}\\
        \hline
        
        Motion artifact reduction for brain MRI & 2018 BA. Duffy, et al \cite{ganHR3d} & GAN with HighRes3dnet as generator \\
        
        Denoise 3D MRI & 2019, M. Ran et al \cite{wganautoencoder} & \textbf{Wasserstein GAN with Convolutional Autoencoder generator} \\
 
    \hline
    \end{tabularx}
    \caption{Overview of studies for noise reduction based in Table 1 from D. Tamada \cite{tamada2020review} (In bold the autoencoder related architecture)}
    \label{table:paper_overview}
\end{table}

    


%%% SECTION NOT SHOWN, examples from tables and algorithms

\iffalse
\section{Latex Examples}

\begin{figure}
	\centering
	\includegraphics[width=0.6\textwidth]{figs/image1.png}
	\caption{Pie de la imagen.}
	\label{fig:context-anoni1}
\end{figure}


\imagen{figs/image1.png}{asdasd}

Un ejemplo de pseudo-código se puede encontrar en el Código \ref{code:RandomSwitch-1}

\begin{algorithm}
	\caption{Pseudocódigo del algoritmo \textit{Random Switch}}
	\label{code:RandomSwitch-1}
	\begin{algorithmic}
		\REQUIRE{El grafo original $G$ y el porcentaje de anonimización $p$ que se desea aplicar.}
		\ENSURE{El grafo $G$ anonimizado.}
		\STATE $num = round(G.num\_edges() * p)$
		\STATE $i = 0$
		\WHILE {$i < num$}
		\STATE {$e_{1} = G.random\_edge()$}
		\STATE $e_{2} = G.random\_edge()$
		\STATE $new\_e_{1} = (e_{1}.origen, e_{2}.origen)$
		\STATE $new\_e_{2} = (e_{1}.destino, e_{2}.destino)$
		\IF {$!G.exist(new\_e_{1})$ \AND $!G.exist(new\_e_{2})$}
		\STATE $G.add\_edge(new\_e_{1})$
		\STATE $G.add\_edge(new\_e_{2})$
		\STATE $G.delete\_edge(e_{1})$
		\STATE $G.delete\_edge(e_{2})$
		\STATE $i=i+1$
		\ENDIF
		\ENDWHILE
		\RETURN $G$
	\end{algorithmic}
\end{algorithm}

Un ejemplo de tabla se puede ver en la Tabla \ref{table:ejemplo_vertex_refi_query}

\begin{table}
	\centering{}
	\begin{tabular}{ l || c | c | l }
		\hline
		Node ID & $\mathcal{H}_{0}$ & $\mathcal{H}_{1}$ & $\mathcal{H}_{2}$ \\
		\hline
		\hline
		Alice & $\epsilon$ & 1 & \{4\}  \\
		\hline
		Bob & $\epsilon$ & 4 & \{1, 1, 4, 4\}  \\
		\hline
		Carol & $\epsilon$ & 1 & \{4\}  \\
		\hline
		Dave & $\epsilon$ & 4 & \{2, 4, 4, 4\}  \\
		\hline
		Ed & $\epsilon$ & 4 & \{2, 4, 4, 4\}  \\
		\hline
		Fred & $\epsilon$ & 2 & \{4, 4\}  \\
		\hline
		Greg & $\epsilon$ & 4 & \{2, 2, 4, 4\}  \\
		\hline
		Harry & $\epsilon$ & 2 & \{4, 4\}  \\
		\hline
	\end{tabular}
	\caption{\textit{Vertex refinement queries}.}
	\label{table:ejemplo_vertex_refi_query}
\end{table}
\fi
\chapter{State of art: related works}
\label{chapter:stateofart}

\section{Overview}
The world relevance and impact of this problem is also shown in the related articles of this subject. The state of art of Deep Learning applied to brain MRI shows the relevance of this field. As we discussed in the introduction, in section \ref{chapter:introduccion}, different deep learning techniques have been used to address the problems derived from brain MRI images: \textbf{classification healthy/disease, tumor segmentation, optimize data acquisition, data augmentation and image enhancement} are the principal ones. 

\begin{tcolorbox}
Nevertheless, we must highlight that all of this problems have common points of works. One of them is the purpose of our project: \textbf{learn MRI representation for reconstruction}.
\end{tcolorbox}

The advance in some of the questions leads to the advance in another. \textbf{Image reconstruction}, which is a mainly sub-problem of image enhancement, could help to achieve better results in:
\begin{itemize}
    \item \textbf{Data acquisition}
    \begin{itemize}
        \item Reconstruct the image from less data collected: faster scanning process \cite{fastmri}.
    \end{itemize}
    
    \item \textbf{Disease detection and segmentation}
    \begin{itemize}
        \item Unsupervised Anomaly Detection: detect diseases with non-labeled data: reconstruction of the disease image differs more than the pathology-free one \cite{pinaya2019}.
        \item Tumor segmentation (widely known as BraTS \cite{brats2014}): encode for extract deep image features and decoder for reconstruction of dense segmentation mask \cite{myronenko20183d}.
    \end{itemize} 
    
    \item \textbf{Data Augmentation}
    \begin{itemize}
        \item Construction of pathology-free image from abnormal and viceversa \cite{2020inpainting} (i.e. lesion inpainting).
        \item Artificial MRI Generation \cite{GanDataAugment2018} \myurl{https://paperswithcode.com/paper/generation-of-3d-brain-mri-using-auto}{\cite{kwon2019gangeneration}} .
    \end{itemize}
    
    \item \textbf{Image Enhancement}
    \begin{itemize}
        \item Reconstruction of cropped parts.
        \item Reconstruction without noise and artifacts \cite{superresolution} \cite{bermudez2018t1autoencoder}, \cite{gondara2016medicalautoencoder}, \cite{wganautoencoder}.
        \item Definition enhancement: from low resolution to high resolution \cite{ganHR3d} \cite{superresolution}.
    \end{itemize}
\end{itemize} 

We want to emphasize the actual relevance of this project. Solving MRI problems using Deep Learning isn't just about how to apply Deep Learning to another field. It is not just a Deep Learning experiment to demonstrate the power of this method. Solving problems with MRI diagnostics (classification, segmentation), MRI quality (MRI enhancement, data augmentation), or MRI acquisition are cutting edge issues in both the field of Computer Science and Healthcare (neuroimaging, neurological analysis, etc.). 



\section{Related works}

We realize this in the overwhelming number of articles using different Deep Learning architectures for solving all kinds of problems with MRI. We will discuss papers addressing different problems but with one similarity: use of image reconstruction in some part of the process (preferably by using autoencoder-based solution). \textbf{However, the main purpose for this project is to apply this reconstruction techniques for noise reduction (image enhancement) and data augmentation (lesion inpainting).}

The main evidence of the big relevance and collaboration between Deep Learning and MR imaging is \textbf{FastMRI by Facebook AI}. In fact, lately, the focus is on \textbf{improving MRI acquisition}, with techniques based on collecting fewer data and using \underline{reconstruction} techniques with Deep Learning with the aim of improving image quality and acquisition speed. The high-impact in the academic field of these kind of studies is based on Facebook AI works. Facebook AI is focused on \textbf{accelerating MR imaging} with AI, and it is his main goal in healthcare nowadays. They created \myurl{https://fastmri.org/}{fastMRI} \cite{fastmri}, a set of models working with some benchmark datasets in order to accelerate the MR imaging acquisition. It is open source, and you can participate in the \myurl{https://fastmri.org/submission_guidelines/}{challenge} with data from New York University. Recently, Facebook and \myurl{https://sites.google.com/view/med-neurips-2020}{NeurIPS} announced that the best models and projects presented for this purpose, even from groups outside Facebook, will be invited to NeurIPS, one of the most important conferences on Neural Information Processing Systems.

%% NVIDIA tumor segmentation
To continue with the different studies using reconstruction methods for distinct purposes, we describe the use of reconstruction for helping \textbf{Tumor Segmentation}. This is another main problem in the state of the art. There is a global academic challenge using labeled brain tumor MRI for BRAin Tumor Segmentation called \textbf{BRATS} \cite{brats2014}. This competition is compound by a MRI dataset from T1, T1c, T2 and FLAIR MRI and the goal is make the segmentation of the distinct parts of the tumor. Using this data as a \myurl{https://paperswithcode.com/task/brain-tumor-segmentation}{benchmark}, lots of different groups are making experiments each year to improve the results. One of these studies using \underline{reconstruction} techniques is the current best outcome for BRAST 2018: \myurl{https://paperswithcode.com/paper/3d-mri-brain-tumor-segmentation-using}{\textbf{A. Myronenko} \cite{myronenko20183d}}. Although their objective is the 3D segmentation of tumors, they use a curious architecture, shown in figure \ref{fig:figs/architecture_myronenko.PNG}, that incorporates an encoder and two decoding branches: one for the creation of tumor segmentation masks and the other for the reconstruction of images. This branch of image reconstruction is only used during training as an additional guide to regularize  the encoder part. The encoder is made by \textbf{ResNet} blocks (Group Norm+ReLu+Conv). The decoder is a \textbf{variational autoencoder} (VAE) made of the distribution layer and deconvolutional upsampling layers with Group Normalization and ReLu.
2 more parts are incorporated in their main loss function for tumor segmentation: Mean square error and Kullback–Leibler divergence of the reconstruction branch.

\imagen{figs/architecture_myronenko.PNG}{Architecture of ResNet-VAE-based network of A. Myronenko. \cite{myronenko20183d}}

%% Classification and pinaya - Simple convolutional autoencoder -semisuepr
Another problem is to \textbf{classify whether an image belongs to a control or a patient}. A recent approach is based on the construction of normative models \cite{marquand2016normative}, and , therefore, the image \underline{reconstruction} based in this normative model. \textbf{Pinaya et al}. \cite{pinaya2019} use this technique to identify abnormal patterns in neuropshychiatric disorders towards achieving \textbf{unsupervised anomaly detection}, so we don't need labeled images from disease data.
Classic methods and approaches based on \textbf{sMRI} (structural magnetic resonance imaging) can't get a good performance in abnormal brain structural detection because neuroanatomical alterations in neurological disorders can be subtle and spatially distributed. Another approach based on Machine Learning methods could improve performance because algorithms are sensitive to these subtle characteristics. The downside of this road is the need for a large amount of image data (control and disease) and that the models are black-boxes with no information on the critical characteristics used for the decision. They developed a \textbf{Deep semi-supervised Autoencoder}, which put unsupervised anomaly detection up for discussion.
The goal of that study is to build an autoencoder which encode the structure of control brains. It means the autoencoder learn the normal distribution for healthy brains and the abnormal MR images would be outliers in that distribution. With this autoencoder defining a distribution for control patients, they define a \textbf{deviation metric} to measure the neuroanatomical deviation in patients. Patients with some disorder should be outliers in this distribution. The architecture and technique used in the experiment is the following:
\begin{itemize}
\item Architecture
    \begin{itemize}
        \item Semi-supervised autoencoder: reconstruction of the image and prediction of age and sex.
        \item 3 hidden layers with SELUs activation function.
        \item Output layer with Linear activation function.
        \item Loss function: MSE from reconstructed and original image + cross-entropy for age + cross-entropy for years + Unsupervised cross-covariance.
        \item 2000 epochs.
        \item ADAM optimizer (adaptive moment estimation) with adaptative learning rate.
        \item 64 samples mini-batches.
    \end{itemize}
\item Transformation of input data:
    \begin{itemize}
        \item Add Gaussian noise to image (0, 0.1).
        \item Feature scaling (normalization).
        \item One-hot encoding for \texttt{sex} and \texttt{age} labels.
    \end{itemize}
\end{itemize}

%% Contar inpaint lessons autoencoder
We continue with a special case: \textbf{lesion inpainting} \cite{2020inpainting}. It can be seen as a \textbf{data augmentation} task (\underline{reconstructing} ‘pathology-free’ versions from patients with any brain disease). In the work of \textbf{José V. Majón et. al.} \cite{2020inpainting} the medial purpose is \textbf{the improvement of the behavior of current brain image analysis pipelines}.  These pipelines are not robust to brain MR images with lesions. For example, a task such as brain part segmentation decreases its accuracy when dealing with lesions. They proposed a \textbf{3D UNet} like network to map the image with lesion to the inpainted image (target). The encoder is made by 3 blocks of a 3D Convolutional Layer with ReLU activation, Batch Normalization and max-pooling. For the decoder they used same architecture but upsampling instead of max-pooling and, in the last step, a  tri-linear interpolation layer followed by a 3D convolution layer (with 8 filters) plus a ReLU and Batch normalization layers for upsampling the image. We can see the diagram pf the architecture in figure \ref{fig:figs/architecture_manjon.PNG}. Everything was trained with MSE loss function. They use lesion masks to generate artificial training data. The use control cases masked out with lesion masks using the software lesionBrain \cite{lesionBrain}.

\imagen{figs/architecture_manjon.PNG}{U-Net based architecture of network of J. V. Manjon et. al. \cite{2020inpainting}}


Besides of all of these principal studies and objectives, there are so many more. \myurl{ https://github.com/TheoEst/
joint_registration_tumor_segmentation}{Théo Estienne et. al.} \cite{otherBraTS2020} in 2020 realize a project based in the study from A. Myronenko \cite{myronenko20183d} which we explained before. They also research Deep Learning architectures for tumor segmentation (BraTS 2018) and use a autoencoder-based network with 2 decoder branches: one for tumor segmentation an another for reconstruction. This last branch is only used for encoder-regularization. They use a fully convolutional VNet architecture, with convolutional layers, ReLU activation function and a intra-block residual
connection with the output of the first activated convolution of the corresponding block. They use also direct connections from encoder to decoder part. 

We discover another study from \myurl{https://paperswithcode.com/paper/a-convolutional-autoencoder-approach-to-learn}{Evan M. Yu et. al. \cite{learnvolrepreCODE}} in which they try to learn volumetric representations from different parts of brain structure. They also use an autoencoder framework.They architecture is composed by 2 components: a spatial transformer network (STN) and a convolutional autoencoder (CAE). The autoencoder is a kind of ResNet-based one, because it uses residual blocks with skip connections, instance normalization and Leaky ReLU activation function.

In order to finalize with the review of the studies of reconstruction applications in MRI, we want to explain one last paper. This work is not focused in MRI, but it achieves very good performance in many tasks like restoration, denoising, super-resolution or image inpainting. \myurl{https://paperswithcode.com/paper/image-restoration-using-convolutional-auto}{XJ. Mao et all \cite{superresolution}} published in 2016 an study about CAE with symetric skip-connections to address those objectives. They also use a residual based network which they call RED-Net. The main characteristics of this network are the skip connections (in which one layer from the encoder are added up to its symmetric layer in the decoder) and the lack of pooling layers. They don't use pooling layers due to pooling discards useful image details that are essential for these tasks. They use MSE loss function. Peak Signal-to-Noise Ratio
(PSNR) and Structural SIMilarity (SSIM) index are calculated for evaluation.

The studies and tasks just explained are the prominent and recent ones but there are other areas and analysis in which MRI restoration could help. Some of them are survival prediction \cite{AnexoReviewAditional}, disease progression \cite{AnexoProgression} or brain connectivity analysis   \cite{AnexoConnectivity}.

\section{Volumes or slices?}
\label{section:soa_vols_slices}

Brain MR images are stored in volumes. It means it are stored as 3D volumes representing somebody's head. Some projects directly use 3D volumes to achieve the purpose (i.e. 3D tumor segmentation or 3D reconstruction). It is more complicated get good results in 3D than in 2D, because results are more relevant in the field. 

When working in 2D we have to consider another decisions. The first one is \textbf{what profile of the volume should we use}. Brain MRI Volume has 3 different views: \textbf{axial} (from above the head), \textbf{sagittal} (from the side of the face, profile) and \textbf{coronal} (from behind the head). We can see the different views in figure \ref{fig:figs/mriviews.jpg}

\imagen{figs/mriviews.jpg}{Left: axial. Middle: sagittal. Right: coronal}

For non-isotropic acquisitions, we should ideally slice them so that the slices are high resolution. For example, if the \textbf{voxel} resolution is 1x1x5 $mm^3$, we should slice the volume so that the slices are 1x1$mm^2$rather than 1x5$mm^2$ (or 5x1$mm^2$). The other issue to address is \textbf{what 2D slices from the volume has relevant information}. Some of the slices are slices of the extremes of the volume, and it didn't represent relevant information about the brain structure-

In this project, we are going to work with 2D slices due to time constraints. Therefore, in this section, we will discuss whether the projects mentioned above have used volumes or images, from which volume profile they have taken the images and how the ones containing important information have been chosen. Our approach will be explained in section \ref{subsubsection:relevantsliceselection}.

A. Myronenko \cite{myronenko20183d} uses 3D volumes for the brain tumor segmentation task (BraTS 2018) with 1x1x1 mm isotropic resolution and size 240x240x155. Consequently, he does not have to get any profile or select slices. Pinaya et. al \cite{pinaya2019} use T1 weighted images, thus 2D slices. They don't say neither the profile used in the images, nor the method to select 2D slices with relevant information. To continue, Manjón et. al \cite{2020inpainting} propose "the first 3D blind inpainting method in medical imaging" as they say. They use the same dataset that us (IXI), they work directly with 3D volumes, so neither profile nor slice election is done. They only preprocessed the volumes in order to normalize the voxels in to 1 mm3 voxel resolution. More recently, Théo Estienne et.al \cite{otherBraTS2020} also address the problem of 3D tumor segmentation of BraTS 2018. Evan M. Yu et. al. \cite{learnvolrepreCODE} uses 3D volumes of OASIS dataset to learn volumetric shape representations for brains structures.

As we can conclude, most projects address 3D volumes because his relevant implications, both in medical field and in deep learning field. However, we have found another articles in which they work with 2D slices. \textbf{C. Bermudez, et al} \cite{bermudez2018t1autoencoder} uses 2D \textbf{axial} slices from BLSA dataset. All subjects were affine-registered to MNIs-space and intensity-normalized before 2D slices are selected. In addition, the only select a \textbf{single midline axial slice} from each volume. Finally they had 528 images with size 220 x 170 voxels. They use this 2D images to build 3 denoiser autoencoders with skip connections (one autoencoder for each level of noise added to train data). The architecture of this 2D denoiser brain mri autoencoder is shown in figure \ref{fig:figs/cbermudezarchitecture.jpg}.

\imagen{figs/cbermudezarchitecture.jpg}{C. Bermudez et. al. Denoising CAE+skip-connections architecture \cite{bermudez2018t1autoencoder}}

\FloatBarrier
\clearpage



\section{Network architectures for images}
%% Contar u-net, resnet VGG, DenseNet, variational

The encoder part of the autoencoder should work as a feature extractor, so we can research the most known Deep Learning Architectures for images in order to use the same architecture or realize transfer learning (with frozen weights or not). We will start explaining the first approach to image processing with Deep Learning, then we will continue with residual network architectures (commonly used in related works) and we will conclude by explaining U-Net and V-Net, the other 2 commonly used networks for medical image segmentation.

\subsection{Alexnet}

First of all, we are going to introduce \textbf{Alexnet} \cite{alexnet}, a deep convolutional neural network for image classification created by Alex Krizhevsky et. al in 2012 which, in that moment, was the best approach for ImageNet classification, improving by 10\% of difference with the second best. It is very important because they apply Deep Learning and convolutional networks to the classification of images and they establish a standard from which new networks are created by adding improvements. The architecture, shown in table \ref{table:alexnet}, is made by 5 convolutional layers and max pooling layer after convolutional 1, 2 and 5 layers, and 3 dense (fully connected) layers as classifier part. Rectified linear unit is used a a activation function (ReLU). Softmax function is used in the last layer in order to represent a probability distribution over the image classes. So AlexNet is characterized by \textbf{convolutional layers, max pooling layers, dense layers as classifier and ReLU activation function}.

Since the moment Alexnet appeared, the improvements made were almost everyone about go deeper in the layers and architecture. However, going deeper didn't solve another problems like vanishing gradient.

\begin{table}[!ht]
  \begin{center}
   \setlength\extrarowheight{2pt} % for a bit of visual "breathing space"
    \rowcolors {2}{gray!15}{}
   \begin{tabular}{c c c c c c c}
    \toprule
    \textbf{Layer} & \textbf{Map} & \textbf{Tensor size} & \textbf{Kernel size} & \textbf{Stride} & \textbf{Activation} & \textbf{Params}\\
    \hline   
    Input Image     & 1     & 227x227x3 & -     & - & -     & - \\
    Convolution 1   & 96    &  55x55x96 & 11x11 & 4 & relu & 34,944 \\
    Max pooling     & 96    &  27x27x96 & 3x3   & 2 & relu & 0 \\
    Convolution 2   & 256   & 27x27x256 & 5x5   & 1 & relu & 614,656 \\
    Max pooling     & 256   & 13x13x256 & 3x3   & 2 & relu & 0 \\
    Convolution 3   & 384   & 13x13x384 & 3x3   & 1 & relu & 885,120 \\
    Convolution 4   & 384   & 13x13x384 & 3x3   & 1 & relu & 1,327,488 \\
    Convolution 5   & 256   & 13x13x256 & 3x3   & 1 & relu & 884,992 \\
    Max pooling     & 256   &  6x6x256  & 3x3   & 2 & relu & 0 \\
    Dense 1         & -     &  4096x1   & -     & - & relu & 37,752,832 \\
    Dense 2         & -     & 4096x1    & -     & - & relu & 16,781,312 \\
    Dense 3         & -     & 4096x1    & -     & - & relu & 4,097,000 \\
    Output Dense    & -     & 1000      & -     & - & Softmax & - \\
    \bottomrule
    \end{tabular}
    \end{center}
    \caption{AlexNet Architecture \cite{alexnet}}
    \label{table:alexnet}
\end{table}

\subsection{Residual networks}

We realize that most of the works of the state of the art  uses residual networks \cite{myronenko20183d} \cite{learnvolrepreCODE} or networks with skip-connections \cite{2020inpainting} \cite{superresolution} \cite{bermudez2018t1autoencoder}, so we will introduce deeper than other architectures.

The emergence of this network rises from deep plain neural networks (as Alexnet) problems on training.  Despite the general belief that the more layers these networks have the more it learns, an experiment made by He et. al. \cite{reslearning} shows that, in some cases, more layers are related to less accuracy even using regularization techniques like L2 or Dropout. Deep plain networks suffer from performance degradation due to the loss of detail in deeper layers and vanishing gradient problem

This residual learning approach is introduced by Kaiming He et. al. in 2016 \cite{reslearning} with the goal of reducing the complexity of deep neural networks. The main innovation of residual networks is that the layers "learn residual functions with reference to the layer inputs, instead of learning unreferenced functions" \cite{reslearning} as they said. This network obtains better performance than before with less complexity in training.

In plain networks, the layers are built to approximate a mapping function from the image to a target: $H(x)$. This is equivalent to approximate the residual of this function: $F(x) = H(x)-x$ and then get the original function as $F(x)+x$. This kind of layer is called \textbf{Residual Block} in which we estimate the residuals and then we add the original input to get our original target function. We can see the architecture of this block in figure \ref{fig:figs/ResidualBlock.png} , which is the picture of the original work. Thus, a \textbf{skip-connection} is include in the network. \textbf{This only consist in adding the input of a stack of layers to the output of this stack of layers}. We have to notice $F(x)$ and $x$ could not have same dimension, so we have to multiply by a linear projection the original input $x$ for matching dimensions. 

\imagen[0.5]{figs/ResidualBlock.png}{Residual Block Architecture \cite{reslearning}}


This paradigm of residual learning gives rise to other ResNet-based architectures, which differs in number of layers and building block compositions. This kind of networks uses residual blocks (there are also several residual block architecture) with skip-connections embedded in some architecture. Two main examples are the ResNet34 and Resnet50 networks, which use different kinds of residual blocks. In addition, He et. al. presents some differentes ResNet architectures which we show in figure \ref{fig:figs/resnet_archs.PNG}. 

In addition, an Autoencoder made with residual blocks made by A. Myronenko \cite{myronenko20183d} is shown in figure \ref{fig:figs/architecture_myronenko.PNG}, in which green blocks represent residual blocks with group normalization. Residual blocks are also used in Yu et. al. \cite{learnvolrepreCODE} work. 


\imagen[1]{figs/resnet_archs.PNG}{Resnet architectures presented in \cite{reslearning}}


\subsection{Other skip-connection-based architectures}

As we explained before, deep plain networks suffer from performance degradation due to the loss of detail in deeper layers and vanishing gradient problem. Residual networks address this problems through residual blocks. But there is a inner idea in ResNets which other architectures also use: \textbf{skip-connections}. In ResNet skip-connections are added in order to estimate the residual function, but these kind of skip-connections can be added to networks with another purpose. Other networks like some Fully Convolutional Networks also use this skip-connections. Other networks like some Fully Convolutional Networks also use this skip-connections. For example, FCN-8 architecture has skip connections, in which some feature maps of the earlier layers are added to later layers.

\textbf{Skip connections from one layer of the encoder to it symmetric layer of the autoencoder} are added to the architecture with 2 purposes. First, to allow the signal to backpropagate straight to the lower layers and thus address the problem of gradient disappearance, facilitating deep network training and, thus, achieving improvements in restoration performance. Second, when we build a deeper network, low-detail of the image could be lost, making deconvolution difficult in recovering task. However, the skip connections pass through the feature maps which carry much image detail and helps deconvolution to recover the original image. 

Some related works use this kind of architecture. C. Bermudez et. al \cite{bermudez2018t1autoencoder} uses skip symmetric connections as can be shown in figure \ref{fig:figs/cbermudezarchitecture.jpg}. This is a convolutional autoencoder with Leaky ReLU, in which the skip-connections add a layer in the encoder to their symmetric layer in the autoencoder (Like FCN but being symmetric).

However, some concrete architectures of CAE + skip-connections have been established due to their good results in some tasks. U-Net \cite{ronneberger2015unet} is one of them, and a U-Net-Autoencoder-based architecture is used by Manjon et. al. \cite{2020inpainting} which we can see in figure \ref{fig:figs/architecture_manjon.PNG}. 




\subsubsection{U-Net and V-Net}

\textbf{U-Net} \cite{ronneberger2015unet} is one of the networks that has skip-connections between symmetric layers. U-Net is a Fully-Convolutional-based Network (FCN) that is mainly used for image segmentation, that's why Manjón \cite{2020inpainting} use a V-Net based architecture for inpainting. The principal 2 differences between a FCN and U-net are the \textbf{symmetry} of U-NET and the skip connections between the downsampling (encoder) path and the upsampling (decoder) path which use \textbf{concatenation} operator instead of sum (sum operator is used in skip-connections in Fully Convolutional Networks).

Each of the 4 blocks used in the downsampling path is made up of 2 convolutional layers (3x3) with batch normalization and ReLU and another 2x2 MaxPooling layer. This block extract advanced features while reducing feature maps sizes. In the upsampling route, the 4 blocks used are made by a 2x2 upconvolutional layer and 2 other convolutional layers like those of the downsampling route to recover size of segmentation maps, the concatenation of the feature map of the symmetric layer of the encoder to give the location information from the encoding path to decoding path and a final 1x1 convolution. The original architecture of U-NET paper is illustrated in figure \ref{fig:figs/unet_architecture.png}. As we can see, this original U-Net architecture is very similar to Manjon et. al. U-Net based autoencoder; see figure \ref{fig:figs/architecture_manjon.PNG}.

\imagen[0.7]{figs/unet_architecture.png}{U-Net architecture of original paper: O Ronneberger et. al. \cite{ronneberger2015unet}}

U-Net has suffered some modifications, one of the most important is \textbf{V-Net}. This network was introduces by F. Miilletari et. al \cite{vnet}. V-Net is used for volumetric biomedical image segmentation and it gets a very good performance in this task. This network is used in the work or T. Estienne \cite{otherBraTS2020} to create a V-Net-based autoencoder to solve Brats 2018 challenge. V-Net combines the skip-symmetric-connections with residual blocks. It means that the encoder and decoder parts are built with residual blocks instead of convolutional blocks as U-Net. V-Net also changes the size of kernels and convolutions in respect to original U-Net, but the main change are the residual blocks. So, V-Net combine the 2 types of skip-connections we have spoken in this research of state of art: residual blocks and larger skip-connections between symetrich layers.


%%SECTION: SUMMARY
\section{Summary of related works}
\label{section:sumary_soa}

We have compiled some recent and prominent works in which they use reconstruction methods for different purposes. We have generally explained their architectures and approaches.

%% Arquitecturas
Although all the collected architectures are based on autoencoders, it has its differences in how the autoencoder is built. First of all, it differs in the main architecture of the blocks of the autoencoder. Different kind of networks like \textbf{ResNet, UNet, VNet, Simple CAE} or \textbf{AlexNet}. Also, in some of them use the Variational Autoencoder approach, even in \cite{myronenko20183d} combine ResNet and VAE. 
Furthermore, there are additional architecture characteristic in which studies differ. 

Regardless of the main architecture, one feature is shared by all related works (works that use Deep Learning for MRI reconstruction): \textbf{skip-connections}. In the works that use ResNet-based \cite{myronenko20183d} \cite{learnvolrepreCODE} architectures, skip-connections are implicit in residual blocks. In this architecture the skip-connection is used for represents the residual estimation function. In U-Net \cite{2020inpainting} and CAE with skip-connections architectures \cite{bermudez2018t1autoencoder} \cite{superresolution}, the skip-connection is used with 2 purposes: allow the signal to backpropagate straight to the lower layers and pass the image details from the convolutional layers to the deconvolutional layers, fighting the low-detail loss \cite{superresolution}. In conclusion with the architecture research, \textbf{the related work use skip connections architectures: ResNet-base and CAE+Skip-Connection-based (U-Net-based and more)}.

%% Contar diferentes loss y medidas: mse, psnr, correlacion, kl
%% Inpainted y super resolution usan MSE, alguno mas tb
The \textbf{loss function} used is also a critical issue. Most of the studies researched use pixel-wise $MSE$, which implicit improves the evaluation metrics $PSNR$ and $SSIM$. In addition, other metrics are also used. $KL divergence$ is added to loss function when VAE is used or $cross-entropy$ and $cross-covariance$ are added when semi-supervised autoencoder is used \cite{pinaya2019}. However, we will discuss in this project the benefits of using $PSNR$ or $SSIM$ directly in loss function. 

%Slices
We are going to work with \textbf{2D slices of 3D brain MRI volumes}. It means that  in our project we are going to reconstruct 2D brain MR images. In order to get 2D images from volumes, we have to get slices, as we can see in image \ref{fig:figs/xyz_slice.PNG}. We can get many 2D images from one brain volume. So, in the  preprocessing step, we firstly must choose \textbf{what brain MRI view we are going to work with}. For non-isotropic acquisitions, we should ideally slice them so that the slices are high resolution: select slices where voxel dimension remains equal for the 2 slice dimensions (i.e. 1x1x5 mm3 should transform into slices of 1x1 mm2 and not 1x5 mm2).

But main step in choosing 2D slices is not that, main step is choose slices which retrieves relevant information. A volume can be seen as a 3D head and some slices (i.e. from the sides) can not retrieve relevant information, because it will retrieve noise or bone parts, but it don't show information about brain structure. We can see this fact in the image \ref{fig:figs/xyz_slice_bad.PNG}, in which is shown the same volume of the image \ref{fig:figs/xyz_slice.PNG} but different slice. In order to get images with relevant information, we have recompiled some main methods in the state of art: \textbf{get fixed number of slices from all volumes, get the middle slice from the volume or develop ourselves a computer vision tool to evaluate the thickness (with opencv)  }. Although the first approach seems the simplest, it is the most used for its good results.

\imagen{figs/xyz_slice.PNG}{2D slices from brain volume IXI ID 002. Different profiles can be seen. Source: myself}

\imagen{figs/xyz_slice_bad.PNG}{2D slices from IXI ID 002. Slices from volume sides with no relevant information. Source: myself}


%% Contar diferentes preprocesados
\textbf{Brain MRI preprocessing} differs depending on the objective. First of all, some of the studies use the images of the dataset directly as the \textbf{target output} of the network. Other works \textbf{enhance the image quality and contrast} before sending it like target output. 
To continue, other preprocessing can be made when sending brain MRI to the input layers. \textbf{Downsampling} the input images could be also useful to reduce the number of parameters of the network. So the resolution of the input image should be balanced between usability (if it is very small is not useful at all) and trainability. We can \textbf{normalize the intensity} of pixels (values from 0 to 1 or mean 0 and standard deviation 1) is very important for a neural network.  Normalization of the inputs helps the training of the network due to several reasons. One of the most important is that a target variable with a large spread of values may result in large error gradient values causing weight values to change dramatically, making the learning process unstable \cite{bishop1995neural}. Besides, \textbf{affine transformations} like translation or rotation could be applied. In addition, we can realize \textbf{data augmentation} in real-time by adding Gaussian noise, other noise or cropping some parts (fixed rectangles or with lesion masks like lessionBrain \cite{lesionBrain}).
  


With this research of the art, we are ready to develop the next stages our approach to brain MRI reconstruction. We emphasize that the paper-discovery techniques used in this research are improved by new frameworks described in section \ref{section:papers_discovery}.

Finally, we made a recompilation of papers reviewed in Table 1 of D. Tamada, 2020 \cite{tamada2020review} and by our own, with some removals and additions based in our goal of the project (see \ref{table:paper_overview}). We have just deeply explained some of them. From the table of D. Tamada we only obtain 1 autoencoder study \cite{bermudez2018t1autoencoder}, 3 sCNN and DnCNN approaches \cite{kidoh2019scnnt1} \cite{ganHR3d} \cite{dncnnnoise2noise} and 1 GAN study.  The other papers have been compiled by our own (Autoencoder based: \cite{pinaya2019} \cite{myronenko20183d} \cite{gondara2016medicalautoencoder} \cite{superresolution} \cite{fuzzyautoencoder} \cite{learnvolrepreCODE} \cite{otherBraTS2020}, GAN-Autoencoder-based: \cite{wganautoencoder}).



\begin{table}[!ht]
    \setlength\extrarowheight{2pt} % for a bit of visual "breathing space"
    \rowcolors {2}{gray!15}{}
    \begin{tabularx}{\textwidth}{C C C}
    \toprule
        \textbf{Purpose} & \textbf{Year, Authors} & \textbf{Network} \\
        \hline
        
        \rowcolor{orange!10}
        \multicolumn{3}{c}{\textbf{Autoencoders}}\\
        
        \hline
        
        Identify brain abnormal structural patterns & 2018, W. Pinaya, et al \cite{pinaya2019} & \textbf{Semi-supervised autoencoder with SeLU and loss MSE+cross-variance} \\
        
        3D Tumor segmentation & 2018, A. Myronenko \cite{myronenko20183d} [Code available] & \textbf{VAE for regularization to encoder (ResNet like)} \\
        
         Lesion inpainting & 2020, J. V. Manjón et al \cite{2020inpainting} & \textbf{3D UNet autoencoder with skip-connections and upsampling at end} \\
         
         General image denoising and super resolution & 2016, XJ. Mao et al \cite{superresolution} [Code available] & \textbf{Residual CAE with symmetric skip connections} \\
        
        Learn Brain volumetric representation &  2018, Evan  M.  Yu  et.   al. \cite{learnvolrepreCODE} [Code available] & \textbf{STN+Residual CAE with skip connections, IN and LReLU}\\
        
        3D Tumor segmentation & 2020, T. Estienne \cite{otherBraTS2020} [Code available] & \textbf{VNET Autoencoder for regularization to encoder.} \\
        
        Denoising for T1 weighted brain MRI & 2018, C. Bermudez, et al \cite{bermudez2018t1autoencoder} & \textbf{Autoencoder with skip connections} \\
        
        Medical image denoise & 2016, L. Gondara, et al \cite{gondara2016medicalautoencoder} &\textbf{Convolutional denoising autoencoder} \\
        
        Brain MRI denoise & 2019, N. Chauhan et al \cite{fuzzyautoencoder} & \textbf{Convolutional denoising autoencoder with Fuzzy Logic filters} \\

        \hline
        \rowcolor{orange!10}
        \multicolumn{3}{c}{\textbf{sCNN and DnCNN}}\\
        \hline

        Denoising for T1, T2 and FLAIR brain images & 2018, M. Kidoh, et al \cite{kidoh2019scnnt1} & Single-scale CNN with DCT \\
        
        Motion artifact reduction for brain MRI & 2018, P. Johnson, et al \cite{scnnmotion} & Single-scale CNN\\
        
        Denoising for multishot DWI & 2020, M Kawamura et al \cite{dncnnnoise2noise} & DnCNN with Noise2Noise \\
        
        \hline
        \rowcolor{orange!10}
        \multicolumn{3}{c}{\textbf{GAN}}\\
        \hline
        
        Motion artifact reduction for brain MRI & 2018 BA. Duffy, et al \cite{ganHR3d} & GAN with HighRes3dnet as generator \\
        
        Denoise 3D MRI & 2019, M. Ran et al \cite{wganautoencoder} & \textbf{Wasserstein GAN with Convolutional Autoencoder generator} \\
 
    \bottomrule
    \end{tabularx}
    \caption{Overview of studies for reconstruction based in Table 1 from D. Tamada \cite{tamada2020review} (In bold the autoencoder related architecture)}
    \label{table:paper_overview}
\end{table}
\FloatBarrier


\section{New paper-discovery frameworks}
\label{section:papers_discovery}

In this state of art research, we have use new techniques and frameworks among the classical ones. \myurl{https://www.connectedpapers.com/}{Connected Papers} is a network science framework to improve the search of papers. There are also other platforms like \myurl{https://paperswithcode.com/}{Papers With Code} and \myurl{https://distill.pub/}{Distill} that improve the experience of article discovering and article visualization-interaction.
 
I will explain some little examples of paper discovery using this tools. \textit{Connected Papers} retrieve us a graph about the relationship of a given paper. A paper is related with another if one is cited by the other. In addition, the graph contains papers citated by the succesors of the main one. So, the graph contains the most important prior works and Derivate works of our main paper, giving us a perfect tool for paper discovery. \myurl{https://www.connectedpapers.com/main/37a18be8c599b781cc28b6a62d8f11e8a6a75169/3D-MRI-brain-tumor-segmentation-using-autoencoder-regularization/graph}{One example} made for A. Myronenko work \cite{myronenko20183d} is shown in figure \ref{fig:figs/connected_papers_myronenko.PNG} 
 
 \imagen{figs/connected_papers_myronenko.PNG}{Graph of connected papers for A. Myronenko 2018 \cite{myronenko20183d}}

\textit{Papers with code} is a web page, in which are stored papers with it official code implementations. It also group works in different subjects of study, like Medical issues, image segmentation, etc. In addition, it recompile the benchmarks for different machine learning tasks, and it make a ranking of the papers (with the code an pdf linked from the page). We show in figure \ref{fig:figs/pwc_myronenko.png} an example of the Myronenko work. We can see the abstract, the link of the paper and multiple links for implementations. Also, we can see the tasks and the benchmark results. There are more thing that we can do and research in this framework. 

\imagen{figs/pwc_myronenko.png}{A. Myronenko work \cite{myronenko20183d} in Papers with code}
\chapter{Scope}
\label{chapter:scope}

In this chapter, we will establish the aims of the project. We have just spoken about the problem to be solved. So now we have to enumerate the concrete objectives of the project. 

\TBD{The objectives will be temporal and will be redefined in further stages.}


\section{Hypothesis}

\textbf{We will build an auto encoder for reconstructing and denoise T1-weighted brain MRI. It will remove noise and will learn the underlying structure of the images in a lower dimensional space, and will reconstruct the image based on this low dimensional space.}

\section{Primary aims}

\begin{itemize}
    \item To build a \textbf{noise-reducer autoencoder} that gets good results with control \textbf{T1-weighted brain MRI}: given a T1-WBMRI, the autoencoder will return the same image as equal as we can to the original, but removing the noise.
    \item Research a good autoencoder architecture and parameters (loss function, batch-norm or not batch-norm, regularization, etc).
    \item Establish a good brain MRI pre-processing.
\end{itemize}

\section{Secondary aims}

\begin{itemize}
    \item Develop the Deep Learning code using one of the most relevant framework, Python, and one of the best-known libraries: \TBD{Tensorflow, Keras, or Pytorch (To Be chosen in further stages)} .
    \item Use an agile methodology: SCRUM. This methodology should be used in the project. We will use the Zenhub tool of Github as a helper in the project management.
\end{itemize}

These next objectives will be addressed if the primary ones are reached. We could see this aims like a extra for the project. If we achieved good performance in this task, we would research about how to apply this solution to disease detection or data augmentation.

\begin{itemize}
    \item Build a semi-supervised autoencoder.
    \item Build a tumor detection system (based on supervised learning or based in the output of the autoencoder \cite{pinaya2019}).
    \item Research GAN architectures for noise and artifact reduction.
\end{itemize}

\chapter{Planning and Methodology}
\label{chapter:planning}

In this chapter, we are going to discuss the scheduling for the project and the methodology used in this one.

\section{Research plan}
In this section, we are making a time planning for our project. Planning a project is a very important feature, because we can manage the time properly and we can keep a realistic task-calendar. For this purpose, we are going to elaborate a \textbf{Gantt Diagram}. This diagram is a very common resource used in project management \cite{tfm_cunha}.

Our diagram is a weekly Gannt Diagram. It has 17 weeks ([mm/dd/yyyy]): 

\begin{itemize}
    \item Week 1: from 09/14/2020 to 09/20/2020
    \item Week 17: from 01/01/2020 to 01/10/2021
\end{itemize}

It is built by all the main tasks that a master's degree final project must have and some personalized ones for this project. So we will have 6 big phases derived from project submits.

\TBD{This diagram is up to the date of October 18, 2020. We show it in the next page.}

\clearpage

% documentation. This reproduces an example from Wikipedia:
% http://en.wikipedia.org/wiki/Gantt_chart
%


\definecolor{barblue}{RGB}{153,204,254}
\definecolor{groupblue}{RGB}{51,102,254}
\definecolor{linkred}{RGB}{165,0,33}
%\renewcommand\sfdefault{phv}
%\renewcommand\mddefault{mc}
%\renewcommand\bfdefault{bc}
\setganttlinklabel{s-s}{START-TO-START}
\setganttlinklabel{f-s}{FINISH-TO-START}
\setganttlinklabel{f-f}{FINISH-TO-FINISH}
\sffamily
\begin{ganttchart}[
    canvas/.append style={fill=none, draw=black!5, line width=.75pt},
    hgrid style/.style={draw=black!5, line width=.75pt},
    vgrid={*1{draw=black!5, line width=.75pt}},
    today=5,
    today rule/.style={
      draw=black!64,
      dash pattern=on 3.5pt off 4.5pt,
      line width=1.5pt
    },
    today label font=\footnotesize\bfseries,
    title/.style={draw=none, fill=none},
    title label font=\bfseries\footnotesize\color{black!70},
    title label node/.append style={below=7pt},
    include title in canvas=false,
    bar label font=\mdseries\footnotesize\color{black!70},
    bar label node/.append style={left=1.2cm},
    bar/.append style={draw=none, fill=black!63},
    bar incomplete/.append style={fill=barblue},
    bar progress label font=\mdseries\scriptsize\color{black!60},
    group incomplete/.append style={fill=groupblue},
    group left shift=0,
    group right shift=0,
    group height=.5,
    group peaks tip position=0,
    group label node/.append style={left=.2cm},
    group progress label font=\bfseries\scriptsize,
    link/.style={-latex, line width=1.5pt, linkred},
    link label font=\scriptsize\bfseries,
    link label node/.append style={below left=-2pt and 0pt}
  ]{1}{17}
  \gantttitle[
    title label node/.append style={below left=7pt and -3pt}
  ]{WEEKS:\quad1}{1}
  \gantttitlelist{2,...,17}{1} \\
  \ganttgroup[progress=100]{Project Selection}{1}{1} \\
  \ganttgroup[progress=99]{Scope and planning}{2}{2} \\
  \ganttbar[
    progress=95,
    name=WBS1A
  ]{Title, Keywords and Abstract}{2}{2} \\
  \ganttbar[
    progress=99,
    name=tit
  ]{Overview, relevance and aims}{2}{2} \\
  \ganttbar[
    progress=100,
    name=over
  ]{Methodology and planning}{2}{2} \\[grid]
  \ganttgroup[progress=95]{State of art research}{3}{5} \\
  \ganttbar[progress=99]{Search bibliography}{3}{5} \\
  \ganttbar[progress=99]{Search similar code projects}{4}{5} \\
  \ganttbar[progress=85]{Resume state of art, redefine aims}{4}{5}\\[grid]
  
  \ganttgroup[progress=0]{Project development}{6}{14} \\
  \ganttbar[progress=0]{Data preparation}{6}{6} \\
  \ganttbar[progress=0]{Deep Learning Design}{6}{7} \\
  \ganttbar[progress=0, name=cod]{Deep Learning codification}{7}{9}\\
  \ganttbar[progress=0, name=exe]{Experiment execution}{10}{12}\\
  \ganttbar[progress=0]{Discuss results, improvements}{12}{13}\\
  \ganttbar[progress=0]{Update documentation}{14}{14}\\[grid]
  
  \ganttgroup[progress=0]{Documentation}{15}{16} \\[grid]
  
  \ganttgroup[progress=0]{Presentation and defence}{17}{17} \\
  \ganttlink[link type=f-s]{cod}{exe}
\end{ganttchart}
\rmfamily


\clearpage


\section{Methodology}

In this section we must choose a common academic Data Mining development methodology, in which there are described the phases, tasks and its relationships.

The description and nature of the project are very helpful at this point because the methodology used in the project will depend on the nature of it. The main characteristic of this project is its research-oriented purpose, so we can label the project as an \textbf{academic research project}. Nevertheless, the main objective of this research is to develop a software component (a Deep Convolutional Autoencoder). We can also describe the project as a \textbf{software project}. In addition, the project is located in the Machine Learning and Deep Learning areas. These areas are very related to Maths, Statistics, and Computer Science. In all of these fields, the aim is to analyze data in a quantitative way. We analyze how the variables are related, how the autoencoder performance with a concrete measure, how it trains getting concrete metrics (how it learns, time, overfitting...), etc. So our methodology should be \textbf{quantitative}. We will take a representative sample of brain MRI, we will train the autoencoder and inference the results to all the population. All this sample and inference techniques are addressed by the validation methods of Machine Learning (Train/test, Cross-validation to reduce bias, etc). 

So, due to the nature of the project, we have to apply a methodology for an \textbf{quantitative academic research project for data mining software development}. 

\begin{tcolorbox}
In a very summarized way, we will start researching the state of art, defining the problem, and proposing a model to solve the target problem. We will choose and prepare our data. Then we will develop the data mining software solution for this problem, evaluating each step. Finally, we will evaluate our model an get a conclusion for our hypothesis. Thus, \textbf{CRISP-DM methodology} embed all of these steps and it will be chosen as the project methodology.
\end{tcolorbox}

The methodology that best suits our project is \textbf{CRISP-DM} \cite{crisp}. The \textit{\textbf{CR}oss-\textbf{I}ndustry \textbf{S}tandard \textbf{P}rocess for \textbf{D}ata \textbf{M}ining} is a framework used for creating and deploying machine learning solutions. Moreover, research and quantitative tasks can be embedded  in the CRISP-DM phases (i.e. state of art research phase can fit into business understanding CRISP-DM phase and quantitative evaluation can fit into model evaluation).

As we know, agile methodologies are often used in software development. CRIPS-DM is neither an agile methodology nor a waterfall one. This methodology has clear stages, but the order of them is not strict and we could move forward and back whenever we need, in order to improve our data mining final model. In fact, this movement between phases is widely used. Also it has a iterative cycle, in which data, data preparation, modelling and evaluation are improved wit the previous iteration feedback.

Figure \ref{fig:figs/crisp_dm.png} shows the phase dependencies and order. As we can see, the straight lines define the dependencies between phases as in a classical methodology. Nevertheless, We can see the circle and the two-arrowed straight lines that show the flexibility and the agile similarity of CRISP-DM.


\imagen[0.5]{figs/crisp_dm.png}{CRISP-DM Cycle}

The phases of CRISP-DM \cite{crisp} are the following (\TBD{In later stages, references will be made to the specific sections of the document where the phases are carried out}):
\begin{description}
    \item[Business Understanding]: deep analysis of the business needs. In this phase we can establish an objective. In our case, we can research the state of art for Deep Convolutional Autoencoder for brain MRI and propose a model based on this research.
    \item[Data Understanding]: we should research the data sources as  \myurl{https://brain-development.org/ixi-dataset/}{IXI}, data quality and we should explore the data and its characteristics.
    \item[Data Preparation]: Data should be cleaned, filtered, selected and integrated if necessary. We could carry out tasks like preprocessing T1 weighted brain MRI or realize data augmentation. I will be explained in-depth in section \ref{section:datalifecycle}.
    \item[Modeling]: Specify the model to use and the architecture, parameters, etc. Maybe running several model architecture and hyper-parameter optimization to reach the most powerful model. So in can be an iterative process.
    \item[Evaluation]: We must evaluate models properly to get meaningful conclusions. There are many techniques of model evaluation and it should be made carefully.
    \item[Deployment - Publication]: As our main goal is academic research, this phase would be \textit{Publication}. The tasks are: review the project and generate the final report.
\end{description}
%\chapter{Theoretical concepts}
\label{chapter:theory}

\section{\TBD{To Be Done}}
%\chapter{Techniques and tools}
\label{chapter:tools}

\section{\TBD{To Be Done}}


\chapter{Project development}
\label{chapter:development}

\TBD{TBD: The text wrote in this section is not official. This text are notes for further stages. At current stage, state of art, we only write the section of the related works }

\section{\TBD{To Be Done}}

\section{\TBD{Data Life-cycle}}
\label{section:datalifecycle}

\begin{itemize}
    \item Capture: download from web.
    \item Storage: Locally.
    \item Preprocessing:
    \begin{itemize}
        \item Cleaning and filtering: clean corrupted data, filter no-needed data.
        \item MRI needed preprocessing (FreeSurfer in \cite{pinaya2019}).
        \item Feature engineering: Normalize, downsampling.
        \item Data augmentation: Noise addition, crop parts, ¿Rotations?
    \end{itemize}
    \item Analysis: create autoencoder
    \item Visualization: Optional - Make some visualizations from results, layers or behavior of the autoencoder.
    \item Publication.
\end{itemize}

\section{\TBD{Data Exploration}}

First notebook: MRI data familiarization and nibabel familiarization. How to mange MRI, etc. Ids, spreadsheet of metadata, voxel sizes, shape sizes.

 \begin{itemize}
     \item 0.9375 x 0.9375 x 1.2 $mm^3  \rightarrow$ 576 volumes.
     \item 0.9766 x 0.9766 x 1.2  $mm^3  \rightarrow$  5 volumes.
 \end{itemize}


\section{\TBD{Data Preprocessing}}
\label{section:data_preprocessing}

\begin{itemize}
    \item Select profile: voxel size
    \item affine-registered to MNIs-space and intensity-normalized
    \item Select slices: relevant information
    \item Clean duplicates
    \item Improve target images: mean local, enhance contrast, etc
    \item Downsampling
    \item Pixel normalization: range [0-1]
    \item Data split: split train and test: stratify by sex, age and etc.
    \item Data augmentation in train
    \begin{itemize}
        \item Add gaussian noise
        \item Crop images
    \end{itemize}
    
\TBD{For non-isotropic acquisitions, we should ideally slice them so that the slices are high resolution. For example, if the voxel resolution is 1x1x5 mm3, we should slice the volume so that the slices are 1x1mm2 rather than 1x5mm2 (or 5x1mm2)
2. Overriding the above, we need to be consistent in which orientation we are slicing. In other words, if we are getting axial slices from one volume, we should make sure we get axial slices for all patients. Otherwise, the network will likely not train well.}    
    
\end{itemize}
\chapter{Conclusion and Outlook}
\label{chapter:Conclusion}

\section{Conclusion}

First of all, we want to highlight the deep research of the state of the art, in which we have made a review of pipelines, methods and applications. We want that chapter to be seen as a review paper, where we have followed the steps that are made in the classical review papers.

The deeper discussion of the results of every experiment is already done along the last section \ref{subsection:experiments} while we made the experiments. We also summarized it in section \ref{section:results}. Therefore, we can conclude that \textbf{the combination of skip-connections and residual building blocks in shallow-autoencoders have significant benefits in the reconstruction of magnetic resonance images of the healthy brain, quantitatively and qualitatively} (see Table \ref{table:expaug} and Figures \ref{fig:figs/daug-mse-qualitative.png}, \ref{fig:figs/daug-dssim-qualitative.png}, \ref{fig:figs/ttest-pvals.png} and \ref{fig:figs/all_test_metrics.png}). We can also obtain other conclusions on other aspects. Every method, regarding every architecture and loss function used, is outstandingly able to remove Gaussian noise and the blur of a brain image. Besides, architectures with skip connections (RES-UNET and skip-connection CAE) and with dropout regularization (Myronenko) fix excellently the dropouted pixels. Although dropout reconstruction is also very good for Shallow residual models, we can see a few amount of dropouted pixels not reconstructed. 

To continue, regarding the reconstruction of blanked-out regions, the proposed method and the shallow residual ones are the best in this task. Besides, reconstructions made with DSSIM loss outperforms the ones made with the methods trained with MSE loss. Methods trained with DSSIM loss are better at the task of reconstructing the real structure and shape. MSE methods are more prominent to predict blurred gray pixels. Finally, the conclusion about the use of L2 regularization that it works better when DSSIM loss is used. Although L2 leads the MSE methods to lightly avoid to reconstruct a regular gray cloud of pixels in blanked-out regions, the effect is more noticeable in DSSIM methods, in which we can see that there is a better approach to simulate the brain structure. This happens because DSSIM tries to reduce difference in the structural information instead of the pixelwise difference.

Finally, as a Master's Thesis on Data Science, we also have focus our project in keep the good habits of Data Science projects. We have follow the Crisp-DM stages and the classical data life-cycle, keeping in mind the good behaviour with the data.


\section{Future work}

Given the timescales of the project and the extension of the report, other planned experiments have been out of the scope of the project. There are three groups of experiments that we wanted to do: experiments of explainable AI (XAI) to explore the behavior of our models, experiments of potential applications of our trained models, and finally experiments with other alternative models and architectures.

The first group of futures experiments are about XAI. For explain the difference of behaviour between our residual CAE, skip-connections CAE and residual U-NET CAE, we will propose a visual exploration of the activation of feature maps of each layer, as it is made in \cite{kernelsactivation} an in other \href{https://towardsdatascience.com/using-skip-connections-to-enhance-denoising-autoencoder-algorithms-849e049c0ac9}{resources}. In addition, we could also explore the latent space visually and with dimensional reduction techniques. With this experiment we would dive deep into the differences in reconstruction between residual and skip connections architectures. Another potential experiment to explore deeper in the qualitative results of our method is analyze the reconstruction of the images in the sides of the brain. As we have included more relevant images than the state of the art projects made, our hypothesis is that our model would reconstruct better the images of the edges of the brain.

The second group of future experiments is related to the potential applications of the trained methods. First, one possible application is the detection of brain disorders. This is based on the Pinaya et. al. research \cite{pinaya2019}. Our autoencoder defines a normative model of the healthy brain structure and the resonances with disorders would be outliers in the distribution. Therefore, a measure distance should be calculated between the input and output, and, if the input image has some disorder, this distance would be higher than a threshold (or maybe we could implement a supervised or unsupervised method with several distance measures). Second, another potential application would be the improvement of brain MRI automatic analysis pipelines. For instance, we can explore the benefits in brain segmentation pipelines. We could use the DeepBrain network to segment our images. We would set the brain mask from the original images made by DeepBrain as the ground truth. Then we calculate the DeepBrain mask from the randomly corrupted input, and also the DeepBrain mask from our reconstruction of the corrupted input. Then we calculate the differences or accuracy between the brain mask from the corrupted image and the ground truth and also between the brain mask from the reconstructed image and the ground truth. Finally, we compare both measures with the hypothesis that the latter accuracy should be higher than the former. So the steps would be: (1) Get mask from original brain as GT (2) get mask from corrupted mri (3) get mask from reconstructed input (4) get metrics ACU(mask, corrupted-mask) and ACU(mask, reconstructed-mask) (5) Compare them.


The last group of experiments includes those related to training different hyperparameter configuration, new models and architectures. First, we could implement deeper models and analyze the effects of using more layers. We could also use transfer learning, using some pretrained network with another set of images (like ResNet-50 trained with CIFAR). Last, but not least, we must use other families of models, which are based on another theoretical concepts instead of dimensionality reduction. We are regarding to use \textbf{generative models}. Generative models are on the edge, becoming more popular in many fields. Therefore, using Generative Adversarial Networks or Variational Autoencoders, could be a proper approach to capture the distribution of the structure of a normal brain. 


% bibliografia
\addcontentsline{toc}{chapter}{Bibliography}
\bibliographystyle{unsrt}
\bibliography{referencias}

\end{document}