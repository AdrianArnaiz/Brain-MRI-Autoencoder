\pagenumbering{roman} 
\setcounter{page}{1} 
\pagestyle{plain}

%%%%%%%%%%%%%%%%
%%% CREDITOS %%%
%%%%%%%%%%%%%%%%
\chapter*{Créditos/Copyright}

\vspace{1cm}

\begin{figure}[ht]
    \centering
	\includegraphics[scale=1]{images/license.png}
\end{figure}

Este obra está bajo una \href{http://creativecommons.org/licenses/by-nc-sa/3.0/es/}{licencia de Creative Commons \textit{Reconocimiento, NoComercial, CompartirIgual} 3.0 España}.

\begin{figure}[ht]
    \centering
	\includegraphics[scale=1]{images/license.png}
\end{figure}

This work is licensed under a \href{http://creativecommons.org/licenses/by-nc-sa/3.0/es/deed.en}{Creative Commons Attribution-NonCommercial-ShareAlike 3.0 Spain License}.

The \myurl{https://github.com/AdrianArnaiz/Brain-MRI-Autoencoder}{official code repository of this Master's Thesis} is licensed under MIT license.

%%%%%%%%%%%%%
%%% FICHA %%%
%%%%%%%%%%%%%
\chapter*{FICHA DEL TRABAJO FINAL}

\begin{table}[ht]
	\centering{}
	\renewcommand{\arraystretch}{2}
	\begin{tabular}{r | l}
		\hline
		Título del trabajo: & \vtop{\hbox{\strut Deep Convolutional Autoencoder} \hbox{\strut for control brain MRI:}\hbox{\strut Development and Applications}}\\
		\hline
        Nombre del autor: & Adrián Arnaiz Rodríguez\\
		\hline
        Nombre del colaborador/a docente: & Baris Kanber\\
		\hline
        Nombre del PRA: & Ferrán Prados Carrasco\\
		\hline
        Fecha de entrega (mm/aaaa): & 01/2021\\
		\hline
        Titulación o programa: & Máster en Ciencia de Datos\\
		\hline
        Área del Trabajo Final: & Area Medicina (TFM-Med)\\
		\hline
        Language: & English\\
		\hline
        Keywords & Deep Learning, Brain MRI, Autoencoder\\
		\hline
	\end{tabular}
\end{table}

%%%%%%%%%%%%%%%%%%%
%%% DEDICATORIA %%%
%%%%%%%%%%%%%%%%%%%
\chapter*{Dedication/Cite}

\TBD{To Be Done.}


%%%%%%%%%%%%%%%%%%%
%%% Agradecimientos %%%
%%%%%%%%%%%%%%%%%%%
\chapter*{Acknowledgment}

\TBD{To Be Done.
Si se considera oportuno, mencionar a las personas, empresas o instituciones que hayan contribuido en la realización de este proyecto.}

%%%%%%%%%%%%%%%%
%%% RESUMEN  %%%
%%%%%%%%%%%%%%%%
\chapter*{\centering Abstract}
\addcontentsline{toc}{chapter}{Abstract}

\onehalfspacing

\begin{quote}
{The analysis of brain MRI is critical for a proper diagnosis and treatment of neurological diseases. Improvements in this field lead to better health quality. Numerous branches can be still enhanced due to the nature of MRI recompilation: disease detection and segmentation, data augmentation, improvement in data collection, or image enhancement are some of them.

For several years, many approaches have been taken to address this. Machine Learning and Deep Learning emerge as very popular approaches to solve problems. Several kinds of data mining problems (supervised, unsupervised, dimension reduction, generative models, etc) and algorithms can be applied to the problem-solving of MRI. Besides, new emerging deep learning architectures for another kind of image task can be helpful. New types of convolution, autoencoders or generative adversarial networks are some of them.

Therefore, the purpose of this work is to apply one of these new techniques to T1 weighted brain MRI. We will develop a Deep Convolutional Autoencoder, which can be used to help some problems with neuroimaging. The input of the Autoencoder will be control T1WMRI and aims to return the same image, with the problematic that, inside its architecture, the image travels through a lower-dimensional space, so the reconstruction of the original image becomes more difficult. Thus, the Autoencoder represents a normative model.

This normative model will define a distribution (or normal range) for the neuroanatomical variability for the illness absence. Once trained with these control images, we will discuss the potential application of the autoencoder like noise reducer or disease detector.}
\end{quote}

%{http://www.ece.cmu.edu/~koopman/essays/abstract.html}
%{http://www.ece.cmu.edu/~koopman/essays/abstract.html}

\textbf{Keywords}: Deep Learning, Brain MRI, Deep Convolutional Autoencoder, Image denoising.

\clearpage

\chapter*{\centering Resumen}

\addcontentsline{toc}{chapter}{Resumen}

\begin{quote}
{El análisis de las resonancias magnéticas cerebrales es fundamental para un diagnóstico y tratamiento adecuados de las enfermedades neurológicas. Se pueden mejorar ámbitos del análisis debido a la naturaleza de la recopilación de resonancias: detección y segmentación de enfermedades, aumento de datos, mejora en la extracción o mejora de imágenes.

El aprendizaje automático y el aprendizaje profundo surgen como nuevas alternativas populares para resolver estos problemas. Se pueden aplicar varios enfoques de minería de datos y algoritmos para la resolución de problemas relacionados con la neuroimagen (supervisados, no supervisados, reducción de dimensionalidad, modelos generativos, etc.). Además, las nuevas arquitecturas emergentes de aprendizaje profundo, desarrolados para otro tipo de tareas de imagen, pueden ser útiles. Algunas de ellas son nuevos tipos de convolución, autoencoders o GAN.

Por lo tanto, el propósito de este trabajo es aplicar una de estas nuevas técnicas a resonancias cerebrales tipo T1. Desarrollaremos un Autoencoder convolucional profundo, que puede usarse para ayudar con algunos problemas de neuroimagen. La entrada del Autoencoder será el imágenes de control T1WMRI y tendrá como objetivo devolver la misma imagen, con la problemática de que, dentro de su arquitectura, la imagen viaja por un espacio de menor dimensión, por lo que la reconstrucción de la imagen original se vuelve más difícil. El autoencoder representa un modelo normativo.

Este modelo normativo definirá una distribución (o rango normal) para la variabilidad neuroanatómica para la ausencia de enfermedad. Una vez entrenado con imágenes de control, discutiremos la aplicación potencial del Autoencoder como reductor de ruido o detector de enfermedades.

}
\end{quote}
\textbf{Keywords}: Aprendizaje profundo, Imágenes cerebrales de resonancias magnéticas, Autoencoder convolucional profundo, eliminación de ruido de imágenes.



\clearpage